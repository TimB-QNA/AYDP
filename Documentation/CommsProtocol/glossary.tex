\chapter{Glossary}
The following terms and abbreviations are used in this document.

\section{Data Types}
Data types are made up of two or three sections. The signedness, type and size. an unsigned integer capable of holding integer values from 0 to 255 would be described as uint8. That is an unsigned, integer of 8 bits in length. The signedness may be omitted for signed types, thus in the previous example, int8 would denote a type capable of storing integer values from -128 to +128. The abbreviations given in table \ref{tab:typeabbrv} may be used to describe the data concerned.

\begin{table}[H]
  \centering
  \begin{tabular}{ c c }
  Abbreviation & Meaning \\
\hline
  int  & integer \\
  flt  & floating point \\
  bool & unsigned integer containing flags \\ 
  \end{tabular}
  \caption{Message Types}
  \label{tab:typeabbrv}
\end{table}