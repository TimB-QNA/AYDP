\chapter{Message Types}
The AYDP protocol supports a number of message types, given below.

\begin{table}[H]
  \centering
  \begin{tabular}{ c c c }
  Message Type & Purpose & Link\\
\hline
   0 & Time sync                & \ref{msg000} \\
   1 & Wind speed and direction & \ref{msg001} \\
   2 & Environmental conditions & \ref{msg002} \\
   3 & Position                 & \ref{msg003} \\
   4 & Position fix quality     & \ref{msg004} \\
   5 & Position fix detail      & \ref{msg005} \\
   6 & Vessel data              & \ref{msg006} \\
   7 & Direct output control    & \ref{msg007} \\
   8 & System value update      & \ref{msg008} \\
   9 & System control message   & \ref{msg009} \\
 100 - 126 & User defined message & \ref{msg-user} \\ 
 128 & Heartbeat                & \ref{msg128} \\
 255 & Reserved                 & \ref{msg255} \\
  \end{tabular}
  \caption{Message Types}
  \label{table:msg:types}
\end{table}

\section{Time synchronisation}
\label{msg000}
The time synchronisation message contains no data and is used to set the time on remote devices, by using the message time in the header. If this message is expidited, then it's meaning changes to that of a heartbeat see \ref{msg127}.

\section{Wind speed and direction}
\label{msg001}
TBD

\section{Environmental conditions}
\label{msg002}
TBD

\section{Position}
\label{msg003}
Position
\begin{table}[H]
  \centering
  \begin{tabular}{ c c c }
  Byte & Description & Type \\
\hline
   0-8  & Latitude  & flt64 \\
   9-16  & Longitude & flt64 \\
   17-21 & Altitude  & flt32 \\
  \end{tabular}
  \caption{Position information}
\end{table}

\section{Position fix quality}
\label{msg004}
Fix Quality
\begin{table}[H]
  \centering
  \begin{tabular}{ c c c }
  Byte & Description & Type \\
\hline
    0  & Fix Type             & uint8  \\
   1-2 & Mean DOP             & uflt16 \\
   3-4 & Horizontal DOP       & uflt16 \\
   5-6 & Vertical DOP         & uflt16 \\
    7  & Number of satellites & uint8  \\
  \end{tabular}
  \caption{Position fix quality information}
\end{table}

\section{Position fix detail}
\label{msg005}
For the fix detail, bytes 1 to 7 are repeated for each Satellite. Thus a message for 10 satellites would be 71 bytes long. 

\begin{table}[H]
  \centering
  \begin{tabular}{ c c c }
  Byte & Description & Type \\
\hline
    0  & Number of satellites               & uint8  \\
    1  & Satellite \#n ID                   & uint8  \\
   2-3 & Satellite \#n Azimuth               & uflt16 \\
   4-5 & Satellite \#n Elevation             & uflt16 \\
   6-7 & Satellite \#n Signal to Noise ratio & uflt16 \\
  \end{tabular}
  \caption{Position fix detail from GPS}
\end{table}

\section{Vessel Data}
\label{msg006}
Time Sync
\begin{table}[H]
  \centering
  \begin{tabular}{ c c c }
  Byte & Description & Type \\
\hline
    0  & Number of satellites               & uint8  \\
    1  & Satellite \#n ID                   & uint8  \\
   2-3 & Satellite \#n Azimuth               & uflt16 \\
   4-5 & Satellite \#n Elevation             & uflt16 \\
   6-7 & Satellite \#n Signal to Noise ratio & uflt16 \\
  \end{tabular}
  \caption{Position fix detail from GPS}
\end{table}

\section{Direct Output Control}
\label{msg007}
The ouput channels can be controlled directly, as long as direct control is enabled (table \ref{tab:svu:prescribed}). This is typically reserved for testing and calibration, or for controlling the platform as one might a radio controlled boat.

The control system has a series of outputs, which may contain channels within them, consequently each address is made up of two integer values (device and channel) and a floating point value for the calibrated position. That is, to move a rudder +10 degrees on device 1, channel 2, the message content for that channel should read:
\begin{verbatim}
0x01 0x02 0x00 0x00 0x20 0x41
\end{verbatim} 

This message type supports setting an arbitrary number of channels, upto the addressable maximum (65,536) by specifying the device, channel and output data for each channel. If a channel is defined more than once, the last value is used as the active output.

\begin{table}[H]
  \centering
  \begin{tabular}{ c c c }
  Byte & Description & Type \\
\hline
    0  & Number of channels & uint16 \\
    1  & Output device      & uint8 \\
    2  & Output channel     & uint8 \\
  3-7  & Calibrated output  & flt32 \\
  \end{tabular}
  \caption{Direct output}
\end{table}

\section{System Value Update}
\label{msg008}
The system value update message allows values within the software (such as control loop targets e.g. ordered heading or waypoints) to be changed on the fly. There are three modes in which a value may be updated, as specified in table \ref{tab:svu:updTypes}.

\begin{table}[H]
  \centering
  \begin{tabular}{ c c }
    Update type           & Value\\
\hline
    Single Value          & 1 \\
    Entire Array          & 2 \\
    Element Within Array  & 3 \\
  \end{tabular}
  \caption{Value update types}
  \label{tab:svu:updTypes}
\end{table}

Each variable which can be changed must be registered at an address. This is typically hard-coded in the software, and therefore will vary dependant on the purpose of the software. The author of the software should give the address mapping for their code in their documentation. Some addresses are prescribed and are required for other messages in this protocol to provide the correct functionality. These are given in table \ref{tab:svu:prescribed}.

The address space is 16 bits wide, giving 65,536 possible addresses which can be edited (some addresses are read-only). These addresses are not memory locations. They are in effect an array of pointers to other memory locations. This means that any address can contain any type of data, or an array of any type of data. The sending and recieving software must know what type of data is expected for each address, and therefore, type data is not included in the message. Only a single address can be updated in each message.

\begin{table}[H]
  \centering
  \begin{tabular}{ c c c }
  Byte & Description & Type \\
\hline
    0   & Update type & uint8 \\
   1-2  & Address     & uint16 \\
   3-6  & Array size or element & uint32 \\
   7-   & Data & As required \\
  \end{tabular}
  \caption{System value update}
\end{table}

Where a single value is edited, the ``array size or element" entry is set to 0. If an entire array is being written, the array in this message over-writes the array in memory. This function must be used with caution, as changing array length may cause the software to crash. The programmer should guard against this possibility.

When writing an array, the array size specifies the number of elements in the array, and the data entry is a contigious list of the array elements.

\begin{table}[H]
  \centering
  \begin{tabular}{ c c c c }
  Address & Description & Type & Access\\
\hline
  0x0000   & Enable direct control & uint8 (0/1) & read-write \\
  0x0001   & System Name & char[40] & read-only \\
  0x0002   & System ID   & char[40] & read-only \\
  0x0003   & Hardware Version & char[40] & read-only \\
  0x0004   & Firmware Version & char[40] & read-only \\
  \end{tabular}
  \caption{Prescribed system addresses}
  \label{tab:svu:prescribed}
\end{table}

For example, to enable direct control the following bytes would be sent:
\begin{verbatim}
0xFF 0x88 0x09 0x00 0x00 0x00 0x00 0x01 CS
\end{verbatim}

\section{System Control Message}
\label{msg009}

The system control message allows the user to control the operation of the remote system, and request data. In some cases, such as with half-duplex communication, this may be preferable to a simple broadcast policy using system value update (\ref{msg008}). The command message consists of a single byte indicating the command, then further bytes specifying the arguments to that command.

\begin{table}[H]
  \centering
  \begin{tabular}{ c c c }
    Command              & Value & Reference \\
\hline
    Request data          & 1 & \ref{scm:rqd} \\
    Store configuration in non-volatile storage  & 2 & \ref{scm:scd}\\
  \end{tabular}
  \caption{System Control Message Commands}
  \label{tab:scm:commands}
\end{table}

For example, to request the data at address 8, the following bytes would be sent:
\begin{verbatim}
0xFF 0x89 0x09 0x00 0x00 0x00 0x01 0x08 0x00 0x7F
\end{verbatim}

While to save the configuration data, the following bytes would be sent:
\begin{verbatim}
0xFF 0x89 0x08 0x00 0x00 0x00 0x02 CS
\end{verbatim}

\subsection{Request Data}
\label{scm:rqd}

\begin{itemize}
\item Parameters
\subitem Address
\end{itemize}

This command allows the user to request data stored at a particular address. The system responds to this command with a system value update (\ref{msg008}) message. The svu message contains the required data, but does not give type information. If an address references an array, the whole array is returned.

\subsection{Store configuration in non-volatile storage}
\label{scm:scd}

\begin{itemize}
\item Parameters
\subitem None
\end{itemize}

This command enables calibration or other configuration data to persist across system restarts. Typically, the remote system will store data in EEProm on microcontrollers or store to disk on larger systems. Issuing this command will save the current value of all non-volatile settings.

\section{User defined messages}
\label{msg-user}
The message type range 100-126 is provided for user-defined messages. The message format should be cearly defined in the user's program documentation with reference to this protocol.

\section{Heartbeat}
\label{msg128}
The heartbeat message is the expedited form of the time synchronisation message (\ref{msg000}). It serves to ensure that communication is still valid.

\section{Reserved Message}
\label{msg255}
The type 255 messages are provided to improve the error rejection properties of the protocol. If this message type is found, the first byte of the stream should be discarded, as the assumption is that the data stream has become a byte out of sync. This can happen in cases where the checksum of the previous message is 0xFF.
